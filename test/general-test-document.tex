\documentclass[a4paper,oneside,11pt]{memoir}

\usepackage{lbt}
\usepackage{kantlipsum}
\usepackage[dvipsnames]{xcolor}
\usepackage{newpxtext}
\usepackage{tcolorbox}
\usepackage{pdfpages}     % \includepdf
\usepackage{caption}
  \captionsetup{labelfont={small,bf,color=blue4},textfont={small,color=blue4},labelsep=quad,margin=10pt}
\usepackage[hidelinks]{hyperref}
\usepackage{cleveref}

\renewcommand{\thefootnote}{\textcolor{blue4}{\arabic{footnote}}}

\lbtLogChannels{all}

\lbtDocumentWideOptions{vector.format = bold}

\lbtDefineLatexMacro{integral=lbt.Math:integral}
\lbtDefineLatexMacro{V=lbt.Math:vector}
\lbtDefineLatexMacro{Vijk=lbt.Math:vectorijk}
\lbtDefineLatexMacro{sm=lbt.Math:simplemath}
\lbtDefineLatexMacro{smallnote=lbt.WS0:smallnote}
\lbtDefineLatexMacro{mathlistand=lbt.Math:mathlistand}   % TODO: add to other tex files
\lbtDefineLatexMacro{mathlistor=lbt.Math:mathlistor}     % TODO: add to other tex files
\lbtDefineLatexMacro{mathlist=lbt.Math:mathlist}   % TODO: add to other tex files
\lbtDefineLatexMacro{mathlistdots=lbt.Math:mathlistdots}   % TODO: add to other tex files
\lbtDefineLatexMacro{mathsum=lbt.Math:mathsum}     % TODO: add to other tex files

\hfuzz=5pt

\setcounter{tocdepth}{2}
\pagestyle{empty}

% ----------------------------------------------------------------------

\begin{document}

\lbtDraftModeOff{}

\tableofcontents


% ----------------------------------------------------------------------
% {{{    Example of some LBT commands
% ----------------------------------------------------------------------

\begin{lbt}
  @META
    TEMPLATE   lbt.Doc.Chapter
    TITLE      LBT general test document
    LABEL      ch-general

  +BODY
    TEXT The \texttt{lbt.Doc.Chapter} template formats a whole chapter. You then use the command \texttt{SECTION} (and so on for subsection, etc.) to divide your document into sections.
    TEXT It is also fine to use the more general \texttt{lbt.Basic} template and use the \texttt{CHAPTER} command. It makes no difference. But it is probably convenient to put only one chapter's worth of content into a single \texttt{lbt} environment.
    SECTION (label) sec-numbered :: A numbered section
    TEXT \kant[2-3]
    SECTION* (label) sec-unnumbered :: A section without a number
    TEXT This section was created using \texttt{SECTION*}.
\end{lbt}

\begin{lbt}
  @META
    TEMPLATE   lbt.Doc.Section
    TITLE      A section inside an LBT document
    LABEL      sec-1

  +BODY
    TEXT This section is written in its own \verb|\begin{lbt}| using the \texttt{SECTION} template. Using the \texttt{LABEL} feature, we can say it is \Cref{sec-1} (in \Cref{ch-general}).
    TEXT .o nopar :: This particular \texttt{TEXT} command has \texttt{.o nopar} set, meaning \verb|\par| will be suppressed after this sentence.
    TEXT (This is another \texttt{TEXT} command.)
    TEXT .o fobar :: Hello.
\end{lbt}

% }}}

\end{document}
